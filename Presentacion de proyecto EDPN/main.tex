\documentclass{beamer}
\usepackage[utf8]{inputenc}
\usepackage{booktabs,siunitx}
\usetheme{Berkeley}
\usecolortheme{beaver}
\newcolumntype{C}{>{\centering\arraybackslash}m{4.8cm}}
\newcolumntype{D}{>{\centering\arraybackslash}m{2.5cm}}
%------------------------------------------------------------
%This block of code defines the information to appear in the
%Title page
\title[?] %optional
{?}

\subtitle{?}

\author[] % (optional)
{Daniel Minaya, Felipe Urrutia, Sebastián Toloza}

\institute[] % (optional)
{
  Departamento de Ingeniería Matemática, \\
  Universidad de Chile
}

\date[?] % (optional)
{?}

% \logo{\includegraphics[height=1cm]{overleaf-logo}}

%End of title page configuration block
%------------------------------------------------------------



%------------------------------------------------------------
%The next block of commands puts the table of contents at the 
%beginning of each section and highlights the current section:

\AtBeginSection[]
{
  \begin{frame}
    \frametitle{Contenidos}
    \tableofcontents[currentsection]
  \end{frame}
}
%------------------------------------------------------------


\begin{document}

%The next statement creates the title page.
\frame{\titlepage}

% Por ahora escribí parte de lo que habíamos mostrado en el video, por lo que de a poco iré complementando.

%---------------------------------------------------------
%This block of code is for the table of contents after
%the title page
\begin{frame}
\frametitle{Contenidos}
\tableofcontents
\end{frame}
%---------------------------------------------------------
\section{?}
\subsection{?}
%---------------------------------------------------------
%Highlighting text

\begin{frame}{Propuesta del proyecto}
    
\end{frame}

\begin{frame}{Marco Teórico}
    
\end{frame}

\begin{frame}
\frametitle{Segmentación de Imágenes: EDP}

Luego, dada una curva inicial $C_0$, la ecuación que se busca resolver para $u(x,t)$ es

\begin{equation*}
    (EDP) = 
    \begin{cases}
    \frac{\partial u}{\partial t} =g(x)|\nabla u| \left( div\left( \frac{\nabla u}{|\nabla u|} \right)+\kappa \right) & \text{en } \Omega \times (0,\infty) \\
    u(x,0) = u_0(x) & \text{en } \Omega
    \end{cases}
\end{equation*}

En donde $u_0(x)$ es una función distancia con signo, dada por 

\begin{equation*}
    u_0(x)=
    \begin{cases}
    d(x,C_0) & \text{si $x$ está \textbf{dentro} de $C_0$}\\
    0 & \text{si $x$ está \textbf{en} $C_0$}\\
    -d(x,C_0)  & \text{si $x$ está \textbf{fuera} de $C_0$}
    \end{cases}
\end{equation*}

\end{frame}

\begin{frame}
\frametitle{Segmentación de Imágenes: EDP}
Y $g(x)$ es una \textit{stopping function}, dada por

\begin{equation*}
    g(x) = \frac{1}{1+|\nabla f_{\sigma}(x)|^2/\lambda^2}
\end{equation*}

En donde $f_{\sigma}$ corresponde a la suavización de la imagen a partir de un kernel gaussiano de desviación estándar $\sigma$ y $\lambda$ es un factor de contraste.

\end{frame}

\begin{frame}{Métodos de Implementación Numérica}

Diferencias Finitas

\begin{equation*}
    \left( I - \tau \sum_{l \in \{x,y\}} A_l(u^n)\right)u^{n+1} = u^n + \tau|\nabla^-u|^n\kappa g
\end{equation*}
    
En donde los coeficientes de $A_l$ vienen dados por
    
\begin{equation*}
    \hat{a}_{ijl}(u^n):=
    \begin{cases}
    a_i|\nabla u|_i^n \frac{2}{\left(\frac{|\nabla u|}{b} \right)_i^n+\left(\frac{|\nabla u|}{b} \right)_j^n} & j \in N_l(i)\\
    -a_i|\nabla u|_i^n \sum\limits_{m \in N_l(i)} \frac{2}{\left(\frac{|\nabla u|}{b} \right)_i^n+\left(\frac{|\nabla u|}{b} \right)_m^n} & j=i\\
    0 & \text{otro caso}
    \end{cases}
\end{equation*}    
    
\end{frame}

\begin{frame}{Métodos de Implementación Numérica}
Y las aproximaciones de $|\nabla u|$ vienen dadas según los siguientes casos:

\medskip

%No supe por ahora cómo hacer que se vea mejor

\begin{enumerate}
    \item Si $\kappa \leq 0$, entonces $|\nabla u|_i^n \approx  |\nabla^- u|_i^n $, que viene dado por
    \begin{equation*}
        |\nabla^- u|_i^n =
        (\max(D^{-x}u_i^n,0)^2 + \min(D^{+x}u_i^n,0)^2 \\ 
         + \max(D^{-y}u_i^n,0)^2 + \min(D^{+y}u_i^n,0)^2 )^{1/2}
    \end{equation*}
    \medskip    
    \item Si $\kappa > 0$, entonces $|\nabla u|_i^n \approx  |\nabla^+ u|_i^n $, que viene dado por
    \begin{equation*}
        |\nabla^- u|_i^n =
        (\min(D^{-x}u_i^n,0)^2 + \max(D^{+x}u_i^n,0)^2 \\ 
         + \min(D^{-y}u_i^n,0)^2 + \max(D^{+y}u_i^n,0)^2 )^{1/2}
    \end{equation*}
    \begin{equation*}
        
    \end{equation*}
\end{enumerate}

\end{frame}

\begin{frame}{Primeros Resultados}
    
\end{frame}

\begin{frame}{Código de Ejecución de los Métodos}
    
Repositorio Github

\medskip

\url{https://github.com/furrutiav/edpn-computer-vision-2022}
    

\end{frame}

\begin{frame}{Dataset para Imágenes}
    
\end{frame}

\begin{frame}{Resultados Avanzados}
    
\end{frame}

\begin{frame}{Proyecciones del Proyecto}
    
\end{frame}

\begin{frame}{Referencias}
    
\end{frame}
%---------------------------------------------------------
\end{document}